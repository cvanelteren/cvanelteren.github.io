% Created 2022-02-27 Sun 08:04
% Intended LaTeX compiler: xelatex
\documentclass[11pt]{article}
\usepackage{graphicx}
\usepackage{longtable}
\usepackage{wrapfig}
\usepackage{rotating}
\usepackage[normalem]{ulem}
\usepackage{amsmath}
\usepackage{amssymb}
\usepackage{capt-of}
\usepackage{hyperref}
\usepackage[english]{babel}
\author{Casper van Elteren}
\date{\today}
\title{}
\begin{document}

\tableofcontents

\section{Installing Debian 11 ``Bullseye'' on the NSA325}
\label{sec:org4127854}
My old zyxal NSA325 has served me well. I paid sub 150 euros
for it and allowed me to store my data securely offsite. The
NSA325v2 released  in 2014 and supports  a 2 bay NAS  with a
1.2Ghz proccessor. I didn't know  it at the time of writing,
but  it   turned  out  that  the   NSA325(v2)  offered  nice
``hackable'' features  with the accessibility of  UART pins on
the motherboard. I can't call  exactly when I bought the NAS
but  it has  served me  for  many years.  When the  NSA325v2
became end of life, I thought about upgrading it to arch arm
but life happened and I  ended up spending my time somewhere
else.

The original firmware  allowed for ssh access,  and some app
support  such as  dropbox, an  internal photo  app. However,
support  for  other  apps  such  as  a  plex  media  server,
self-hosted cloud such as owncloud or nextcloud was minor at
best. The  opensource community, however, were  able to hack
the device a  little through the fonz plug  which allowed to
install third  party apps such as  owncloud, sickbeard, plex
media server  and so on. This  pluging served me well  as my
needs for the NAS was mainly for storage, and the bare-bones
interface from Zyxal  was enough to setup  an apache server,
with sickbeard  pulling my shows  from the web.  Support for
the fonz plug diminished over  the years as the market share
for the zyxal NAS became end of life.

Luckily, in the opensource community people have developed a
way  to   boot  into  another  file   system  all  together,
completely replacing the original  firmware and file system.
This would prevent my old NAS  to become ewaste and allow it
to start a  second life. The process  involved replacing the
original  bootloader such  that it  could boot  into another
linux distribution. This  is not for the faint  of heart and
requires some ``low-level'' access  to ensure the process goes
correctly  in case  stuff  goes wrong  (and  it most  likely
will). I looked into this process many years ago, and nearly
ended up bricking my device. As the disk space slowed filled
on my trusty NAS, I decided this year to upgrade my storage,
and in the process upgrade the software to the modern age.

The  process  involved  (1) replacing  the  bootloader,  (2)
installing a new OS (Debian). Initially, I wanted to install
arch arm.  But unfortunately, support for  the kirkwood cpus
was dropped February 2022 due  to lack of NEO floating point
operations on  these devices.  Luckily, Debian 11  was still
supported. I  hope that this  support will cary  through for
another few years  before eventually, I need  to upgrade the
NAS completely. That point hasn't  reached yet, so let's get
started. This post  is mostly for me to  organize my thought
on the  process and  is not necessarily  a tutorial.  I will
provide resources that I  used. Debugging took several hours
in order for me to boot  into Debian and in the end required
UART access to solve completely.

\section{What is uboot?}
\label{sec:orgc5766ea}
Uboot  is  a  bootloader  popular  in  embedded  systems.  A
bootloader is software  that sets up the device  on boot. On
boot  the device  gets  information from  the  BIOS on  what
hardware  is available.  The  bootloader is  then called  to
start launching the  file system. Uboot operates  as a ferry
man ensuring  that the correct drive  is read and the  os is
loaded.

NSA325v2 is  part of the  ARM5 family which is  getting more
and more obsolete. Luckily, running contemporary software is
made   possible   by   a   person  named   Bohdi   over   on
\url{https://forum.doozan.com/index.php}   who    provides   uboot
firmware and corresponding releases for debian.

\section{Upgrading uboot}
\label{sec:org1b6d336}
The instructions are a little involved and for me went wrong
initially. Having access to a  serial monitor (see below) is
essential if one aims to follow in my footsteps. The install
procedure  may require  replacing original  firmware on  the
NAND flash,  which effectively  upgrades (and  removes) the
original firmware. The process is well described but not for
the faint of heart, the relevant links are:

\begin{itemize}
\item \href{https://forum.doozan.com/read.php?2,12096}{Linux kernel and debian rootfs}
\item \href{https://forum.doozan.com/read.php?3,12381}{Uboot install
instructions}
\end{itemize}

At  the time  of writing  the  latest version  of uboot  was
released in 2017 which allows to boot from different devices
including USB sticks.  For the NSA325(v2) the  front USB3 is
disabled at boot, but the back  USB allows to store the file
system on a pen drive, maximizing the storage on the drives.

For  me the  procedure went  wrong, and  my NSA325  was left
blinking  with a  yellow  indicator light.  This means  that
uboot was  not able to  boot, and the  device was left  in a
confused state. I had to obtain  UART access to see what was
going on.

\section{Obtaining UART access}
\label{sec:org2ba666d}
A  particular  welcoming  feature   on  the  NSA325  is  the
availability of  UART pins  on the motherboard.  Opening the
case  up was  a  bit tricky.  There are  two  screws on  the
bottom, after which the shell ``slides'' to the front, showing
access to  the metal casing  with 4 more  additional screws.
After  this,  the motherboard  can  be  freed (and  the  fan
replaced if needed).  I hooked up an FTDI  programmer on the
pins (see figure). We find  two rows, where the ``longer'' row
has access to  3.3V, TX, RX, (empty),  Ground. After hooking
up the FTDI controller, it shows logs on the boot process.

\begin{verbatim}
           +----+----+
           |    |    |
 +----+----+----+----+----+
 |3.3V| TX | RX |    | GND|
 +----+----+----+    +----+
\end{verbatim}

I hooked up  TX and RX with a FTDI  programmer ensuring that
the pins between 5V and ground were set on the programmer.

\section{Debugging}
\label{sec:orga127f21}
After obtaining  UART access,  I was  able to  determine the
cause of the error. My filesystem was corrupt, and the `dts`
file was  not pointing  to the  correct file.  Flashing file
systems on storage has always been a bit strange to me. Most
of the time,  things go as planned.  However, sometimes, the
storage device does not want  to cooperate and your drive is
unreadable even  after reformatting it. Using  gparted I was
able  to  create and  EXT3  partition  labeled rootfs  (per
instruction), and uboot was able to boot in to debian 11.
\section{Installing openmediavault}
\label{sec:org0a5ea09}
After  replacing   thet  stock  OS,  the   accompanying  web
interface also had to be removed. The stock interface showed
some  information about  the  device with  toggles for  File
Transfer Protocol (FPT), Secure  Shell (SSH), IP address and
so on.

\href{./figures/nsa325old.jpg}{NSA admin image}

There are  different linux based opensource  NAS tools, e.g.
FreeNAS, openWRT, openmediavault (OMV), BSDN just to name a few. I
opted to  install OMV which provides  disk images
but more  importantly is  installable on  Debian. Installing
OMV   was  not   straight   forward.  I   started
installing  OMV  through  the  instructions  listed  on  the
website. They provide some shell commands that would add OMV
repo and keyrings to  the Debian package manager (aptitude),
download the  necessary files  and setup OMV.  After running
the commands I was unfortunately greeted with

\begin{verbatim}
armel is not supported
\end{verbatim}

The  kirkwood  CPUs  are  ancient.  The  NSA325v2  I  bought
originates from around 2012 (or  abouts) which makes it nearly
10 years old(!). The main problem was that some dependencies
needed  to be  recompiled for  ARM5. From  what I  gathered,
there  are some  PHP related  files  that are  not any  more
supported for  the ARM5 architecture. Luckily,  there was an
obscure reference  to a repository which  supposedly had the
files  needed to  get PHP  up  and running.  The address  is
\url{https://repozytorium.mati75.eu}. After adding the repo with

\begin{minted}[frame=lines,linenos=true]{bash}
echo "deb http://repozytorium.mati75.eu/debian (lsb -r) main contrib non-free" | tee /etc/aptitude/sources.list.d/mati75.list
apt update

cat <<EOF >> /etc/apt/sources.list.d/openmediavault.list
deb http://packages.openmediavault.org/public shaitan main
# deb http://downloads.sourceforge.net/project/openmediavault/packages shaitan main
## Uncomment the following line to add software from the proposed repository.
# deb http://packages.openmediavault.org/public shaitan-proposed main
# deb http://downloads.sourceforge.net/project/openmediavault/packages shaitan-proposed main
## This software is not part of OpenMediaVault, but is offered by third-party
## developers as a service to OpenMediaVault users.
# deb http://packages.openmediavault.org/public shaitan partner
# deb http://downloads.sourceforge.net/project/openmediavault/packages shaitan partner
EOF

export LANG=C.UTF-8
export DEBIAN_FRONTEND=noninteractive
export APT_LISTCHANGES_FRONTEND=none
apt-get install --yes gnupg
wget -O "/etc/apt/trusted.gpg.d/openmediavault-archive-keyring.asc" https://packages.openmediavault.org/public/archive.key
apt-key add "/etc/apt/trusted.gpg.d/openmediavault-archive-keyring.asc"
apt-get update
apt-get --yes --auto-remove --show-upgraded \
    --allow-downgrades --allow-change-held-packages \
    --no-install-recommends \
    --option DPkg::Options::="--force-confdef" \
    --option DPkg::Options::="--force-confold" \
    install openmediavault-keyring openmediavault

# Populate the database.
omv-confdbadm populate

# Display the login information.
cat /etc/issue
\end{minted}

OMV  installed,  and  was   running.  This  is  great  news!
Confirming  that   the  related  services  were   running  I
installed `systemd` and checked

\begin{minted}[frame=lines,linenos=true]{bash}
root@debian:~# systemctl status openmediavault-engined
*openmediavault-engined.service - The OpenMediaVault engine daemon that processes the RPC request
     Loaded: loaded (/lib/systemd/system/openmediavault-engined.service; enabled; vendor preset: enabled)
     Active: active (running) since Sun 2022-02-27 07:28:22 CET; 21min ago
    Process: 593 ExecStart=/usr/sbin/omv-engined (code=exited, status=0/SUCCESS)
   Main PID: 627 (omv-engined)
        CPU: 3min 26.833s
     CGroup: /system.slice/openmediavault-engined.service
             `-627 omv-engined
\end{minted}

This meant that  OMV was running at  the ipaddress listening
on port 80. Navigating to the  web page I was greeted with a
404  error. More  fixing was  ahead.

After browsing  autocomplete with  `omv[tab]` I  noticed omv
provides a script called `omv-firstaid`. Hitting enter I was
greeted   with  an   interface  allowing   me  to   fix  the
web-interface, how convenient! Thinking  the end was in sight
I quickly ran the command and confirmed the port was 80.

\href{./figures/omv-firstaid.png}{omv-firstaid screen}

The output greeted me with

\begin{minted}[frame=lines,linenos=true]{shell}
Updating web control panel settings. Please wait ...
{'response': None, 'error': {'code': 0, 'message': "Failed to execute command 'export PATH=/bin:/sbin:/usr/bin:/usr/sbin:/usr/local/bin:/usr/local/sbin; export LANG=C.UTF-8; omv-salt deploy run --no-color monit 2>&1' with exit code '1': debian:\n    Data failed to compile:\n----------\n    Rendering SLS 'base:omv.deploy.monit.default' failed: Jinja variable 'salt.utils.templates.AliasedLoader object' has no attribute 'omv_conf.get'", 'trace': "OMV\\ExecException: Failed to execute command 'export PATH=/bin:/sbin:/usr/bin:/usr/sbin:/usr/local/bin:/usr/local/sbin; export LANG=C.UTF-8; omv-salt deploy run --no-color monit 2>&1' with exit code '1': debian:\n    Data failed to compile:\n----------\n    Rendering SLS 'base:omv.deploy.monit.default' failed: Jinja variable 'salt.utils.templates.AliasedLoader object' has no attribute 'omv_conf.get' in /usr/share/php/openmediavault/system/process.inc:196\nStack trace:\n#0 /usr/share/php/openmediavault/engine/module/serviceabstract.inc(62): OMV\\System\\Process->execute()\n#1 /usr/share/openmediavault/engined/rpc/config.inc(167): OMV\\Engine\\Module\\ServiceAbstract->deploy()\n#2 [internal function]: Engined\\Rpc\\Config->applyChanges(Array, Array)\n#3 /usr/share/php/openmediavault/rpc/serviceabstract.inc(123): call_user_func_array(Array, Array)\n#4 /usr/share/php/openmediavault/rpc/rpc.inc(86): OMV\\Rpc\\ServiceAbstract->callMethod('applyChanges', Array, Array)\n#5 /usr/sbin/omv-engined(537): OMV\\Rpc\\Rpc::call('Config', 'applyChanges', Array, Array, 1)\n#6 {main}", 'http_status_code': 500}}
ERROR: Failed to execute command 'export PATH=/bin:/sbin:/usr/bin:/usr/sbin:/usr/local/bin:/usr/local/sbin; export LANG=C.UTF-8; omv-salt deploy run --no-color monit 2>&1' with exit code '1': debian:
Data failed to compile:
----------
Rendering SLS 'base:omv.deploy.monit.default' failed: Jinja variable 'salt.utils.templates.AliasedLoader object' has no attribute 'omv_conf.get'
\end{minted}

Not knowing  exactly what  I was looking,  I decided  it was
time to contact the good people  at omv forums. In the end I
am not sure  exactly what fixed the omv  installation. I can
only recount the coarse steps I took.

\begin{itemize}
\item Purged and removed apache2 [didn't work]. This was because
at first I aimed to run nextCloud. However, this interface
didn't offer me  what I wanted. Omv used `nginx`  as a web
server so I believe it interfered with the package.
\item Purged and reinstalled omv [didn't work]
\item Manually  installed  `salt-common` and  `salt-api`.  After
this point  the error went  away above. The  web interface
was available (yay!), but it wouldn't allow me to sign in.
That is,  after signing  in with  the default  password, I
could  see  with `journalctl  -f`  that  the password  was
correct, yet in the browser I was booted back to the login
screen. The output read
\begin{minted}[frame=lines,linenos=true]{bash}
   Feb 26 20:45:18 debian openmediavault-webgui[8507]: Authorized login from 192.168.1.245 [username=admin, user-agent=Mozilla/5.0 (X11; Linux x86_64) AppleWebKit/537.36 (KHTML, like Gecko) Chrome/98.0.4758.102 Safari/537.36]
    Feb 26 20:45:21 debian postfix/postdrop[637]: warning: mail_queue_enter: create file maildrop/146156.637: Permission denied
\end{minted}
\item As an  aside I would  also like to mention  that `postfix`
had  similar permissions  errors. I  ended up  fixing this
similar to the PHP  issue; setting the correct permissions
o particular  folders for  the mail\textsubscript{queue}. After  a reboot
these errors went away. But  I still couldn't login to the
web-interface.
\item I  think what  finally fixed  it was  the PHP  session not
having    the    proper     write    permissions.    Under
`/var/lib/php/sesssions` there  were files created  by the
user openmediavault. Changing the ownership of that folder
to  the   proper  settings   granted  me  access   to  the
webinterface. SUCCESS!
\end{itemize}

After  this,  I was  greeted  with  a new  shiny  up-to-date
interface for 2022.

\href{./figures/omv-greeting.png}{Great success}





\section{Summary}
\label{sec:org13ad662}
\begin{itemize}
\item Read the instructions completely
\item Be careful when flashing stuff to NAND
\item Check  with  `e2label`  whether   the  drive  is  labeled
correctly by `gparted`
\end{itemize}
\end{document}
